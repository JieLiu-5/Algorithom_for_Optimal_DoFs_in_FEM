\documentclass[review,3p]{elsarticle}

\usepackage{lineno,hyperref}
\modulolinenumbers[5]
\usepackage{subcaption}             % used in subtable
\usepackage{amsmath,amsfonts,amsthm}            % for subequations

\usepackage{mathtools,amssymb}          % for \leqslant
\newcommand{\ddn}[2]{\frac{\mathrm{d}}{\mathrm{d}#1}#2}
\newcommand{\ddt}{\frac{\mathrm{d}}{\mathrm{d}t}}


\usepackage{upgreek}
\usepackage[dvipsnames]{xcolor}
\usepackage{soul}
\usepackage{multirow}

\usepackage{array}
\newcolumntype{C}[1]{>{\centering\let\newline\\\arraybackslash\hspace{0pt}}m{#1}} 
\newcolumntype{L}[1]{>{\raggedright\let\newline\\\arraybackslash\hspace{0pt}}m{#1}} 
\newcolumntype{R}[1]{>{\raggedleft\let\newline\\\arraybackslash\hspace{0pt}}m{#1}} 

\usepackage{booktabs}       % http://ctan.org/pkg/booktabs

\newcommand{\tabitem}{~~\llap{\textbullet}~~}           % for items inside a table
\usepackage{makecell}       % used inside a table
\usepackage{pbox}           % for weak form 3
\usepackage{empheq}
\newcommand*\widefbox[1]{\fbox{\hspace{2em}#1\hspace{2em}}}


\usepackage{colortbl}
\usepackage{esvect}
\usepackage{spreadtab}
\usepackage{numprint}
\usepackage{xstring}
\renewcommand*{\thefootnote}{\fnsymbol{footnote}}
\usepackage[symbol]{footmisc}

\usepackage{siunitx}

\makeatletter       % for rom in deal.ii symbol
\newcommand*{\rom}[1]{\expandafter\@slowromancap\romannumeral #1@}
\makeatother

\usepackage{enumitem}

\usepackage{cleveref}
\crefformat{section}{\S#2#1#3} % see manual of cleveref, section 8.2.1
\crefformat{subsection}{\S#2#1#3}
\crefformat{subsubsection}{\S#2#1#3}

\captionsetup[figure]{labelfont={bf},name={Fig.},labelsep=period}
\captionsetup[table]{labelfont={bf},name={Table},labelsep=space}

\usepackage[labelformat=simple]{subcaption}	        	% order of subfigure with brackets
\renewcommand\thesubfigure{(\alph{subfigure})}
\renewcommand\thesubtable{(\alph{subtable})}


\usepackage{graphicx}
\usepackage{wrapfig}
\usepackage{lipsum}

\usepackage{pgfplots}       % for tikzpicture

\pdfsuppresswarningpagegroup=1      % eliminate warning 'multiple pdfs with page group included in a single page'

\usepackage[ruled,linesnumbered]{algorithm2e}		% for algorithm


\begin{document}

\section{$\alpha_{\rm R}$ for different types of equations}         \label{discretization_error_bench_pois_diff_Helm}

We consider the benchmark Poisson equation, i.e. $u=e^{- (x-1/2)^2}$, but we extend it to diffusion and Helmholtz equations by choosing different $d(x)$ and $r(x)$, see Table~\ref{setting_equation_types} for the settings. The variables in $d(x)$ and $r(x)$ are denoted as $c_d$ and $c_h$, respectively, for which the values of the former ranges from 1 to $10^4$, and that of the latter is equal to 1.

\begin{table}[!ht]
\caption [sss] {Settings of the Poisson, diffusion and Helmholtz equations.}		% \footnotemark
\label{setting_equation_types} 
\centering
 \begin{tabular}{|C{2.5cm}|C{4.2cm}|C{3.8cm}|C{4cm}|} \hline   
{} & {``Poisson''} & {``diffusion''} & {``Helmholtz''} \\ \hline
{$d(x)$} & {$1$} & $1+0.5\sin(c_dx)$ & $1+0.5\sin(x)$  \\	\hline
{$r(x)$} & {0} & 0 & $c_h$ \\	\hline
% {$f(x)$} & {$-e^{- (x-1/2)^2} \left({4x^2 - 4x -1} \right)$}  & $-2 \pi \cos (2 \pi x) + 4 {\pi}^2 \sin (2 \pi x)(x+1)$ & 0 \\ \hline
% {$\|f(x)\|_2$} & {1.60} & {42.99} & {0.00} \\	\hline
% \multirow{2}{2cm}{\centering Boundary conditions} & {$u(0) = e^{-1/4}$} & $u(0)=0$& $u (0) = 1$ \\	\cline{2-4}
% &$u(1) = e^{-1/4}$ & $u_x(1)=2 \pi$  &$ u_x(1) = 0$ \\	\hline
% Analytical solution $u_{\text{exc}}$ & {$e^{- (x-1/2)^2}$} & $\sin (2 \pi x)$ & $a e^{(1+i) x} + (1-a) e^{-i x}$, $a=1/{((1-i) e^{1+2i}+1)}$ \\	\hline
% {$\|u_{\text{exc}}\|_2$} & {0.92} & 0.71 & 1.26 \\	\hline
\end{tabular}
\end{table}


The offset $\alpha_{\rm R}$ for the three types of equations using the standard FEM ($P_2$ elements) and the mixed FEM ($P_4/P_3^{\rm disc}$ elements) are shown in 2 writing 1 1d summary of offsets.
%Fig.~\ref{alpha_R_equation_type}. 
% Not that, for the Helmholtz equation, only $c_h=1$ is considered.

%\begin{figure}[!ht]
%\hspace{0.2cm}
%% \centering
%\begin{subfigure}{7.5cm}
%    \includegraphics[width=1.0\linewidth]{alpha_R_equation_type_SM.png}
%    \caption{The standard FEM.}
%    \label{Fig:alpha_R_equation_type_SM}   
%\end{subfigure}
%\hspace{0.7cm}
%\begin{subfigure}{7.5cm}
%    \includegraphics[width=1.0\linewidth]{alpha_R_equation_type_MM.png}
%    \caption{The mixed FEM.}
%    \label{Fig:alpha_R_equation_type_MM}   
%\end{subfigure}
%\caption{$\alpha_{\rm R}$ for different types of equations.}
%\label{alpha_R_equation_type}
%\end{figure}

It shows that the offset $\alpha_{\rm R}$ is generally not affected by the types of equations considered.
However, if we increase $c_h$ for the Helmholtz equation, $\alpha_{\rm R}$ for the solution and first derivative would decrease, see Fig.~\ref{py_oned_helm_p_exp_minus_x_minus_0p5_square_d_1_plus_0p5sinx_SM_error_normal_coeff_variant}.
%\newpage

\begin{figure}[!ht]
    \begin{subfigure}{5.5cm}
        \includegraphics[width=1.0\linewidth]{py_oned_helm_p_exp_minus_x_minus_0p5_square_d_1_plus_0p5sinx_SM_error_normal_coeff_variant_solu.pdf}
        \caption{Solution}
        \label{py_oned_helm_p_exp_minus_x_minus_0p5_square_d_1_plus_0p5sinx_SM_error_normal_coeff_variant_solu}
    \end{subfigure}
    \hspace{-0.2cm}
    \begin{subfigure}{5.5cm}
        \includegraphics[width=1.0\linewidth]{py_oned_helm_p_exp_minus_x_minus_0p5_square_d_1_plus_0p5sinx_SM_error_normal_coeff_variant_grad.pdf}
        \caption{First derivative}
        \label{py_oned_helm_p_exp_minus_x_minus_0p5_square_d_1_plus_0p5sinx_SM_error_normal_coeff_variant_grad}
    \end{subfigure}
    \hspace{-0.2cm}
    \begin{subfigure}{5.5cm}
        \includegraphics[width=1.0\linewidth]{py_oned_helm_p_exp_minus_x_minus_0p5_square_d_1_plus_0p5sinx_SM_error_normal_coeff_variant_2ndd.pdf}
        \caption{Second derivative}
        \label{py_oned_helm_p_exp_minus_x_minus_0p5_square_d_1_plus_0p5sinx_SM_error_normal_coeff_variant_2ndd}
    \end{subfigure}
\caption{Absolute errors for the Helmholtz equation using the standard FEM for different $c_h$.}
\label{py_oned_helm_p_exp_minus_x_minus_0p5_square_d_1_plus_0p5sinx_SM_error_normal_coeff_variant}
\end{figure}

% \newpage
% 
% \bibliographystyle{unsrt}  
% \bibliography{mybibfile}  %%% Remove comment to use the external .bib file (using bibtex).


\end{document}
