% TU Delft beamer template
% Author: Erwin Walraven (initial version was created by Maarten Abbink)
% Delft Universiy of Technology

\documentclass{beamer}
%\usepackage[english]{babel}
%\usepackage{calc}
%\usepackage[absolute,overlay]{textpos}
%\usepackage{graphicx}
%\usepackage{subfig}
%\usepackage{amsmath}
%\usepackage{amsfonts}
%\usepackage{amsthm}
%\usepackage{mathtools}
%\usepackage{comment}
%\usepackage{MnSymbol,wasysym}


%\setbeamertemplate{navigation symbols}{} % remove navigation symbols
%\mode<presentation>{\usetheme{tud}}
\usetheme{Madrid}
\usecolortheme{beaver}


\usepackage{multirow}
\usepackage{makecell}
\renewcommand{\footnotesize}{\fontsize{7pt}{9pt}\selectfont}

\usepackage{tikzsymbols}

\setbeamertemplate{caption}[numbered]

\usepackage{subfig}

\makeatletter       % for rom in deal.ii symbol
\newcommand*{\rom}[1]{\expandafter\@slowromancap\romannumeral #1@}
\makeatother

\usepackage[utf8]{inputenc}
\usepackage{lmodern}


\title[]{Report on the 2D Paper}
\institute[]{Delft University of Technology, the Netherlands}
\author{Jie Liu}
%\date{}

\begin{document}
{
\setbeamertemplate{footline}{\usebeamertemplate*{minimal footline}}
\frame{\titlepage}
}


\section{Aim of the paper}
\begin{frame}{Aim of the paper}
\vspace{-8em}
\begin{enumerate}
 \item To determine $\alpha_{\rm R}$ and $\beta_{\rm R}$ for different FEM methods of different FEM packages for various 2D second-order problems.
 \item To choose FEM methods/elements that give smaller round-off error, i.e. $\alpha_{\rm R}$ and $\beta_{\rm R}$.
 \item To apply the strategy in the 1D paper to find the optimal number of DoFs of 2D problems$^{*}$.
\end{enumerate}
\end{frame}

\section{Problem statement}
\begin{frame}
\frametitle{Problem statement}
\vspace{-7em}
\begin{block}{Problem to be solved}
\scriptsize
\begin{equation}
 - \nabla \cdot (d(x,y) \nabla u) + \mathbf{a}(x,y) \cdot \nabla u + r(x,y) u = f,\qquad (x,y) \in \Omega = [0,\,1] \times [0,\,1],
 \label{problem_to_be_investigated}
\end{equation}
where $d(x,y)$, $\mathbf{a}(x,y)$ and $r(x,y)$ are scalar/vector coefficient functions. The dependent variable $u$ and coefficients can be either real-valued or complex-valued if not stated otherwise. By choosing different coefficient functions, we can have Poisson, diffusion or Helmholtz problems.
\end{block}
\end{frame}


\begin{frame}{FEM Status}
\vspace{-5em}
The status of the application of FEM methods on various Eq.~(\ref{problem_to_be_investigated}) is shown in Table~\ref{table_status_fem_application}.
\begin{table}[!ht]
\scriptsize
\begin{tabular}{ l | c | c}
 & deal.\rom{2} & FEniCS \\ \hline
 Standard FEM ($P_p$) & $\Smiley$\footnote{Working well.} & $\Smiley$ \\ \hline
 Mixed FEM ($RT_p/P_{p}^{\rm disc}$) & $\Smiley$ & -- \\ \hline
 Mixed FEM ($BDM_p/Q_{p-1}^{\rm disc}$)\footnote{The notation $Q$ is in opposition to that of deal.\rom{2}.} & $\Smiley$ & $\Smiley$
\end{tabular}
\caption{Status of application of FEM methods. The element degree $p$ can be of different order if not stated otherwise.}
\label{table_status_fem_application}
\end{table}
\end{frame}

\section{Progress}
\begin{frame}{Progress}
\vspace{-10.5em}
\begin{itemize}
 \item Coefficients for the first derivative in Eq.~(\ref{problem_to_be_investigated}), i.e. $a(x,y)$ and $b(x,y)$ considered for problems with real-valued solutions using the standard FEM. The solution using the mixed FEM is not correct by now, since the function space involved with $u$ is not satisfied. 
\end{itemize}
\end{frame}

%\section{Discussion}
%\begin{frame}{Discussion}
%\vspace{-9em}
%\begin{enumerate}
% \item
%\end{enumerate}
%\end{frame}

\section{Future work}
\begin{frame}{Future work}
\vspace{-10em}
\begin{itemize}
 \item To clarify the existence of the derivative of interest.
 \item To only show the error of $H(div)$ for the second derivative using the mixed FEM.
\end{itemize}
\end{frame}

\bibliographystyle{unsrt}
\bibliography{bibfile_presentation}

\end{document}
